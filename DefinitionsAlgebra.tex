\documentclass[11pt,a4paper]{article}
\author{Дедов Иван}

\usepackage[utf8]{inputenc}
\usepackage[russian]{babel}
\usepackage[OT1]{fontenc}
\usepackage{amsmath}
\usepackage{amsfonts}
\usepackage{amssymb}
\usepackage[left=2cm,right=2cm,top=2cm,bottom=2cm]{geometry}
\usepackage[colorlinks=true,urlcolor=blue]{hyperref}
\usepackage{xcolor}

\newcommand{\N}{\mathbb{N}}
\newcommand{\Z}{\mathbb{Z}}
\newcommand{\Q}{\mathbb{Q}}
\newcommand{\R}{\mathbb{R}}
\renewcommand{\C}{\mathbb{C}}
\newcommand{\F}{\mathbb{F}}

\renewcommand{\a}{\mathbf{a}}
\renewcommand{\b}{\mathbf{b}}
\newcommand{\e}{\mathbf{e}}
\newcommand{\eprime}{\mathbf{e'}}
\renewcommand{\f}{\mathbf{f}}
\newcommand{\A}{\mathcal{A}}
\newcommand{\B}{\mathcal{B}}
\newcommand{\E}{\mathcal{E}}

\newcommand{\I}{\imath}
\newcommand{\J}{\jmath}

\newcommand{\Rg}[1]{\mathrm{Rg}\hspace{0.5mm}#1}
\newcommand{\Ker}[1]{\mathrm{Ker}\hspace{0.5mm}#1}
\renewcommand{\Im}[1]{\mathrm{Im}\hspace{0.5mm}#1}
\newcommand{\Dim}[1]{\mathrm{dim}\hspace{0.5mm}#1}
\newcommand{\Char}[1]{\mathrm{char}\hspace{0.5mm}#1}
\newcommand{\Sgn}[1]{\mathrm{sgn}\hspace{0.5mm}#1}
\newcommand{\Ord}[1]{\mathrm{ord}\hspace{0.5mm}#1}
\newcommand{\Inn}[1]{\mathrm{Inn}\hspace{0.5mm}#1}
\newcommand{\Gr}[1]{\mathrm{Gr}\left(#1\right)}

\newcommand{\vect}[1]{\overrightarrow{#1}}
\renewcommand{\mid}{\hspace{2mm}\middle|\hspace{2mm}}

\setlength{\parskip}{1em}
\setlength{\parindent}{0pt}

\usepackage{fancyhdr}
\pagestyle{fancy}
\fancyhf{}
\fancyhead[L]{\textsf{\textbf{Коллоквиум.} Определения}}
\fancyhead[R]{\textsf{\href{https://t.me/dedov_ivan}{Дедов Иван}, БПИ-206}}
\fancyfoot[C]{\textsf{Высшая школа экономики, ОП <<Программная инженерия>>}}

\renewcommand{\headrulewidth}{0.3pt}
\renewcommand{\footrulewidth}{0.3pt}

\begin{document}

\begin{center}

\begin{huge}
\textsf{Линейная алгебра\\1 курс}
\end{huge}

\vspace{5mm}

\begin{LARGE}
\textsf{\textbf{Материалы для подготовки к коллоквиуму\\ \vspace{3mm}
Определения}}
\end{LARGE}

\end{center}

\rule{\linewidth}{0.3mm}

\vspace{1mm}
\begin{center}
\begin{LARGE}
\textsf{1 модуль}
\end{LARGE}
\end{center}
\vspace{1mm}

\textbf{1. Дать определение умножения матриц. Коммутативна ли эта операция? Ответ пояснить.\\}
Рассмотрим матрицы $A$ размера $n \times p$ и $B$ размера $p \times k$. Тогда матрица $C$ размера $n \times k$, где $$\forall i = \overline{1, n}, \forall j = \overline{1, k}: c_{ij} = \sum_{l=1}^{p} a_{il} \cdot b_{lj},$$ является произведением матриц $A$ и $B$.\\
Эта операция не коммутативна. Рассмотрим матрицы
$$ A =
\left( \begin{matrix}
1 & 1 \\
0 & 0
\end{matrix} \right)
\quad \text{и} \quad
B =
\left( \begin{matrix}
0 & 0 \\
1 & 1
\end{matrix} \right).$$
Тогда
$$ A \cdot B =
\left( \begin{matrix}
1 & 1 \\
0 & 0
\end{matrix} \right) = A,
\quad
B \cdot A =
\left( \begin{matrix}
0 & 0 \\
1 & 1
\end{matrix} \right) = B,
\quad
A \neq B.$$

\textbf{2. Дать определения ступенчатого вида матрицы и канонического (улучшенного ступенчатого) вида матрицы.\\}
Матрица имеет ступенчатый вид, если номера столбцов первых ненулевых элементов всех строк (которые называются ведущими) образуют возрастающую последовательность, а все нулевые строки расположены в нижней части матрицы.\\
Матрица имеет улучшенный ступенчатый (канонический) вид, если она имеет ступенчатый вид, все ведущие элементы равны 1, и в столбце с ведущим элементом все остальные элементы равны 0.

\textbf{3. Перечислить элементарные преобразования строк матрицы.\\}
1) умножение $i$-й строки матрицы на число $\alpha \neq 0$;\\
2) перестановка двух строк в матрице;\\
3) добавление к $i$-й строке матрицы её $k$-й строки с коэффициентом $\alpha$.

\textbf{4. Сформулировать теорему о методе Гаусса (алгоритм приводить не нужно).\\}
Любую конечную матрицу можно элементарными преобразованиями привести к ступенчатому (и каноническому) виду.

\textbf{5. Дать определения перестановки и подстановки.\\}
Всякое расположение чисел $1, \hdots, n$ в определённом порядке называется перестановкой.\\
Подстановкой называется взаимно однозначное (биективное) отображение чисел $1, \hdots, n$ в себя.

\textbf{6. Дать определения знака и чётности подстановки.\\}
Знак подстановки $\Sgn ( \alpha ) = (-1)^N$, где $N$ -- число инверсий в ней.\\
Подстановка $\alpha$ называется чётной, если $\Sgn ( \alpha ) = 1$, а иначе -- нечётной.
\pagebreak

\textbf{7. Выписать общую формулу для вычисления определителя произвольного порядка.}
$$\det A = \sum_{ \sigma \in \mathbf{S}_n } \Sgn ( \sigma ) \cdot a_{1 \sigma(1)} \cdot a_{2 \sigma(2)} \cdot \hdots \cdot a_{n \sigma(n)}$$

\textbf{8. Что такое алгебраическое дополнение?\\}
Алгебраическим дополнением элемента $a_{ij}$ называется число $(-1)^{i+j} M_{ij}$, где $M_{ij}$ -- дополняющий минор элемента $a_{ij}$.

\textbf{9. Выписать формулы для разложения определителя по строке и по 
столбцу.\\}
\textit{Разложение по строке:} Для любой фиксированной строки $i$ справедливо, что $$\det A = \sum_{j = 1}^n a_{ij} A_{ij}.$$\\
\textit{Разложение по столбцу:} Для любого фиксированного столбца $j$ справедливо, что $$\det A = \sum_{i = 1}^n a_{ij} A_{ij}.$$

\textbf{10. Что такое фальшивое разложение?\\}
Для $k \neq i$ верно, что $\sum_{j = 1}^n a_{ij} A_{kj} = 0$.\\
Для $k \neq j$ верно, что $\sum_{i = 1}^n a_{ij} A_{ik} = 0$.

\textbf{11. Выписать формулы Крамера для квадратной матрицы произвольного порядка. Когда с их помощью можно найти решение СЛАУ?\\}
Если $Ax = b$ -- совместная СЛАУ, то $x_{i} \det A = \Delta_i$, где $\Delta_i$ -- определитель матрицы, в которой на месте $i$-го столбца стоит столбец $b$ правых частей уравнений.\\
Отсюда следует, что $$\forall i = \overline{1, n}: x_i = \frac{\Delta_i}{\det A}.$$\\
Решение СЛАУ можно найти тогда, когда $\det A \neq 0$.

\textbf{12. Дать определение союзной матрицы.\\}
Союзной (присоединённой) называется матрица
$$\tilde{A} = \left( \begin{matrix}
A_{11} & A_{12} & \hdots & A_{1n} \\
A_{21} & A_{22} & \hdots & A_{2n} \\
\vdots & \vdots & \ddots & \vdots \\
A_{n1} & A_{n2} & \hdots & A_{nn}
\end{matrix} \right)^T,$$
где $A_{ij}$ -- алгебраическое дополнение элемента $a_{ij}$.

\textbf{13. Дать определение обратной матрицы. Сформулировать критерий её существования.\\}
Обратной к квадратной матрице $A$ называется матрица $A^{-1}$, такая что $A \cdot A^{-1} = A^{-1} \cdot A = E$.\\
Для матрицы $A$ существует обратная $A^{-1}$ $\Leftrightarrow$ $\det A \neq 0$.

\textbf{14. Выписать формулу для нахождения обратной матрицы.\\}
Для квадратной матрицы $A$, такой что $\det A \neq 0$, обратной является матрица $A^{-1} = \frac{1}{\det A} \cdot \tilde{A}$.

\textbf{15. Дать определение минора.\\}
Минором $k$-го порядка матрицы $A$ называется определитель матрицы, составленной из элементов, стоящих на пересечении произвольных $k$ строк и $k$ столбцов матрицы $A$. Обозначение: $M_{i_1 i_2 \hdots i_k}^{j_1 j_2 \hdots j_k}$.
\pagebreak

\textbf{16. Дать определение базисного минора. Какие строки называются базисными?\\}
Базисным называется любой минор, порядок которого равен рангу.\\
Строки, попавшие в базисный минор, называются базисными.

\textbf{17. Дать определение ранга матрицы.\\}
Рангом матрицы называется наивысший порядок отличного от нуля минора.

\textbf{18. Дать определение линейной комбинации строк. Что такое нетривиальная линейная комбинация?\\}
Линейной комбинацией строк $a_1, \hdots, a_s$ одинаковой длины называется выражение вида $\alpha_1 a_1 + \hdots + \alpha_s a_s = \sum_{k = 1}^s \alpha_k a_k$, где $\alpha_1, \hdots, \alpha_s$ -- некоторые числа.\\
Нетривиальной называется линейная комбинация, где среди чисел $\alpha_1, \hdots, \alpha_s$ найдётся $\alpha_i \neq 0$.

\textbf{19. Дать определение линейной зависимости строк матрицы.\\}
Строки $a_1, \hdots, a_s$ называют линейно зависимыми, если существуют такие числа $\alpha_1, \hdots, \alpha_s$, не все равные нулю, что $\alpha_1 a_1 + \hdots + \alpha_s a_s = 0$.

\textbf{20. Дать определение линейно независимых столбцов матрицы.\\}
Если равенство $\alpha_1 a_1 + \hdots + \alpha_s a_s = 0$ выполнено только в случае, когда $\alpha_1 = \hdots = \alpha_s = 0$ (т.е. тривиальной линейной комбинации), то столбцы $a_1, \hdots, a_s$ называются линейно независимыми.

\textbf{21. Сформулировать критерий линейной зависимости.\\}
$a_1, \hdots, a_s$ линейно зависимы $\Leftrightarrow$ хотя бы один из $a_1, \hdots, a_s$ линейно выражается через остальные.

\textbf{22. Сформулировать теорему о базисном миноре.\\}
1) Базисные строки (столбцы), соответствующие любому базисному минору $M$ матрицы $A$, линейно независимы.\\
2) Строки (столбцы) матрицы $A$, не входящие в $M$, являются линейной комбинацией базисных.

\textbf{23. Сформулировать теорему о ранге матрицы.\\}
Ранг матрицы равен максимальному числу её линейно независимых строк (столбцов).

\textbf{24. Сформулировать критерий невырожденности квадратной матрицы.\\}
Рассмотрим квадратную матрицу $A \in \mathbf{M}_n( \R )$. Следующие условия эквивалентны:\\
1) $\det A \neq 0$, т.е. матрица невырождена;\\
2) $\Rg A = n$;\\
3) все строки $A$ линейно независимы.

\textbf{25. Сформулировать теорему Кронекера--Капелли.\\}
СЛАУ $Ax = b$ совместна $\Leftrightarrow$ $\Rg A = \Rg (A \vert b)$.

\rule{\linewidth}{0.3mm}

\vspace{1mm}
\begin{center}
\begin{LARGE}
\textsf{2 модуль}
\end{LARGE}
\end{center}
\vspace{1mm}

\textbf{1. Сформулируйте теорему о структуре общего решения однородной СЛАУ.\\}
Пусть $\Phi_1, \hdots, \Phi_k$ -- ФСР ОСЛАУ $Ax = 0$. Тогда любые решения этой СЛАУ можно представить в виде $x = c_1 \Phi_1 + c_2 \Phi_2 + \hdots + c_k \Phi_k$, где $c_1, \hdots, c_k$ -- некоторые числа.

\textbf{2. Сформулируйте теорему о структуре общего решения неоднородной системы линейных алгебраических уравнений.\\}
Пусть известно частное решение $\tilde{x}$ СЛАУ $Ax = b$. Тогда любое решение этой СЛАУ может быть представлено в виде $x = \tilde{x} + c_1 \Phi_1 + c_2 \Phi_2 + \hdots + c_k \Phi_k$, где $c_1, \hdots, c_k$ -- некоторые числа, а $\Phi_1, \hdots, \Phi_k$ -- ФСР соответствующей однородной СЛАУ $Ax = 0$.
\pagebreak

\textbf{3. Дайте определение векторного произведения векторов в трёхмерном пространстве.\\}
Вектор $\vect{c}$ называют векторным произведением векторов $\vect{a}$ и $\vect{b}$, если:\\
1) $| \vect{c} | = | \vect{a} | \cdot | \vect{b} | \cdot \sin \varphi$, где $\varphi$ -- угол между векторами $\vect{a}$ и $\vect{b}$;\\
2) $\vect{c} \perp \vect{a}, \vect{c} \perp \vect{b}$;\\
3) тройка $\vect{a}, \vect{b}, \vect{c}$ -- правая.

\textbf{4. Сформулируйте три алгебраических свойства векторного произведения.\\}
1) $\vect{a} \times \vect{b} = - \vect{b} \times \vect{a}$ (антикоммутативность);\\
2) $\left( \alpha \vect{a} \right) \times \vect{b} = \alpha \left( \vect{a} \times \vect{b} \right)$;\\
3) $\left( \vect{a} + \vect{b} \right) \times \vect{c} = \vect{a} \times \vect{c} + \vect{b} \times \vect{c}$ (дистрибутивность).

\textbf{5. Выпишите формулу для вычисления векторного произведения в координатах, заданных в ортонормированном базисе.\\}
Пусть $\vect{\I}, \vect{\J}, \vect{k}$ -- правый ортонормированный базис, $\vect{a} = a_x \vect{\I} + a_y \vect{\J} + a_z \vect{k}$, $\vect{b} = b_x \vect{\I} + b_y \vect{\J} + b_z \vect{k}$. Тогда $$\vect{a} \times \vect{b} =
\begin{vmatrix}
\vect{\I} & \vect{\J} & \vect{k} \\
a_x & a_y & a_z \\
b_x & b_y & b_z
\end{vmatrix} =
\vect{\I} \left( a_y b_z - a_z b_y \right) + \vect{\J} \left( a_z b_x - a_x b_z \right) + \vect{k} \left( a_x b_y - a_y b_x \right).$$

\textbf{6. Дайте определение смешанного произведения векторов. Как вычислить объём тетраэдра с помощью смешанного произведения?\\}
Смешанным произведением векторов $\vect{a}, \vect{b}, \vect{c}$ называют число $\left( \vect{a} \times \vect{b}, \vect{c} \right)$.\\
Объём тетраэдра, построенного на векторах $\vect{a}, \vect{b}, \vect{c}$, можно вычислить как $\frac{1}{6} \left\lvert \left\langle \vect{a}, \vect{b}, \vect{c} \right\rangle \right\rvert$.

\textbf{7. Выпишите формулу для вычисления смешанного произведения в координатах, заданных в ортонормированном базисе.\\}
Пусть $\vect{\I}, \vect{\J}, \vect{k}$ -- правый ортонормированный базис, $\vect{a} = a_x \vect{\I} + a_y \vect{\J} + a_z \vect{k}$, $\vect{b} = b_x \vect{\I} + b_y \vect{\J} + b_z \vect{k}$, $\vect{c} = c_x \vect{\I} + c_y \vect{\J} + c_z \vect{k}$. Тогда
$$\left\langle \vect{a}, \vect{b}, \vect{c} \right\rangle =
\begin{vmatrix}
a_x & a_y & a_z \\
b_x & b_y & b_z \\
c_x & c_y & c_z
\end{vmatrix}.$$

\textbf{8. Сформулируйте критерий компланарности трёх векторов с помощью смешанного произведения.\\}
Векторы $\vect{a}, \vect{b}, \vect{c}$ компланарны $\Leftrightarrow$ $\left\langle \vect{a}, \vect{b}, \vect{c} \right\rangle = 0$.

\textbf{9. Дайте определение прямоугольной декартовой системы координат.\\}
Прямоугольной декартовой системой координат назывется пара, состоящая из точки $O$ и ортонормированного базиса $\vect{\I}, \vect{\J}, \vect{k}$.

\textbf{10. Что такое уравнение поверхности и его геометрический образ?\\}
Уравнение $F(x, y, z) = 0$ называется уравнением поверхности $S$, если этому уравнению удовлетворяют координаты любой точки, лежащей на этой поверхности и не удовлетворяют координаты ни одной точки, не лежащей на поверхности.\\
При этом поверхность $S$ называют геометрическим образом уравнения $F(x, y, z) = 0$.

\textbf{11. Сформулируйте теорему о том, что задаёт любое линейное уравнение на координаты точки в трёхмерном пространстве.\\}
Любая плоскость в пространстве определяется уравнением $Ax + By + Cz + D = 0$, где $A, B, C, D$ -- некоторые числа, и любое уравнение вида $Ax + By + Cz + D = 0$, где $A^2 + B^2 + C^2 > 0$, определяет плоскость.
\pagebreak

\textbf{12. Что такое нормаль плоскости?\\}
Если плоскость задана уравнением $Ax + By + Cz + D = 0$, то вектор $\vect{n} = (A, B, C)$ перпендикулярен этой плоскости и называется нормалью к этой плоскости.

\textbf{13. Выпишите формулу для расстояния от точки до плоскости.\\}
Если $\pi$ -- плоскость, заданная уравнением $Ax + By + Cz + D = 0$, а точка $M$ имеет координаты $(x_0, y_0, z_0)$, то расстояние от этой точки до плоскости вычисляется как
$$\rho ( \pi, M ) = \frac{ \left\vert A x_0 + B y_0 + C z_0 + D \right\vert }{\sqrt{A^2 + B^2 + C^2}}.$$

\textbf{14. Общие уравнения прямой. Векторное уравнение прямой. Параметрические и канонические уравнения прямой.\\}
1) Общие уравнения:
$$\begin{cases}
A_1 x + B_1 y+ C_1 z + D_1 = 0 \\
A_2 x + B_2 y+ C_2 z + D_2 = 0
\end{cases}$$
2) Векторное уравнение: $\vect{r} = \vect{r_0} + t \vect{a}$\\
3) Параметрические уравнения:
Если $x_0, y_0, z_0, m, n, k$ -- заданные числа, то прямая задаётся как
$$\begin{cases}
x - x_0 = tm \\
y - y_0 = tn \\
z - z_0 = tk
\end{cases} \quad (t \in \R)$$
4) Канонические уравнения:
$$\frac{x - x_0}{m} = \frac{y - y_0}{n} = \frac{z - z_0}{k}$$

\textbf{15. Сформулируйте критерий принадлежности двух прямых одной плоскости.\\}
Пусть прямые $\ell_1$ и $\ell_2$ заданы своими направляющими векторами $s_1$ и $s_2$, а точки $M_1$ и $M_2$ лежат на прямых $\ell_1$ и $\ell_2$ соответственно. Тогда эти прямые лежат в одной плоскости, если $\left\langle \vect{M_1 M_2}, \vect{s_1}, \vect{s_2} \right\rangle = 0$.

\textbf{16. Выпишите формулу для вычисления расстояния от точки до прямой.\\}
Пусть задана точка $M_1$ с координатами $(x_1, y_1, z_1)$, а прямая $\ell$ задана своим направляющим вектором $\vect{s}$, и $M_0 \in \ell$. Тогда расстояние между точкой $M_1$ и прямой $\ell$ вычисляется как
$$\rho ( M_1, \ell ) = \frac{ \left\vert \vect{M_0 M_1} \times \vect{s} \right\vert }{ \left\vert \vect{s} \right\vert }.$$

\textbf{17. Выпишите формулу для вычисления расстояния между двумя скрещивающимися прямыми.\\}
Если скрещивающиеся прямые $\ell_1$ и $\ell_2$ заданы своими направляющими векторами $\vect{s_1}$ и $\vect{s_2}$, точки $M_1$ и $M_2$ лежат на прямых $\ell_1$ и $\ell_2$ соответственно, то расстояние между ними вычисляется как
$$\rho ( \ell_1, \ell_2 ) = \frac{ \left\vert \left\langle \vect{s_1}, \vect{s_2}, \vect{M_1 M_2} \right\rangle \right\vert }{ \left\vert \vect{s_1} \times \vect{s_2} \right\vert }.$$

\textbf{18. Что такое алгебраическая и тригонометрическая формы записи комплексного числа?\\}
Запись комплексного числа $z$ в алгебраической форме: $z = x + iy$.\\
Запись комплексного числа $z$ в тригонометрической форме: $z = r ( \cos \varphi + i \sin \varphi )$.
\pagebreak

\textbf{19. Дайте определения модуля и аргумента комплексного числа. Что такое главное значение аргумента комплексного числа?\\}
Модулем комплексного числа называется расстояние от $z$ до начала координат (т.е. длина радиус-вектора $z$).\\
Аргументом комплексного числа называется угол между радиус-вектором $z$ и положительным направлением вещественной оси.\\
Главное значение аргумента комплексного числа -- это такое значение аргумента $\varphi$, которое лежит в промежутке $[0; 2\pi)$.

\textbf{20. Сложение, умножение комплексных чисел. Что происходит с аргументами и модулями комплексных чисел при умножении и при делении?\\}
Пусть $z_1 = x_1 + i y_1 = r_1 ( \cos \varphi_1 + i \sin \varphi_1 )$, $z_2 = x_2 + i y_2 = r_2 ( \cos \varphi_2 + i \sin \varphi_2 )$. Тогда:
$$z_1 + z_2 = (x_1 + x_2) + i ( y_1 + y_2 )$$
$$z_1 \cdot z_2 = ( x_1 x_2 - y_1 y_2 ) + i (x_1 y_2 + x_2 y_1 )$$
$$z_1 \cdot z_2 = r_1 r_2 ( \cos ( \varphi_1 + \varphi_2 ) + i \sin ( \varphi_1 + \varphi_2 ) )$$
$$\frac{z_1}{z_2} = \frac{r_1}{r_2}( \cos ( \varphi_1 - \varphi_2 ) + i \sin ( \varphi_1 - \varphi_2 ) )$$

\textbf{21. Что такое комплексное сопряжение? Как можно делить комплексные числа в алгебраической форме?\\}
Если комплексное число $z = x + iy$, то комплексно-сопряжённым с $z$ называется число $\overline{z} = x - iy$. При этом: $$\frac{z_1}{z_2} = \frac{z_1 \overline{z_2} }{ \left\vert z_2 \right\vert ^2 }.$$

\textbf{22. Выпишите формулу Муавра.\\}
Пусть $z = r ( \cos \varphi + i \sin \varphi )$. Тогда $$z^n = r^n ( \cos (n \varphi) + i \sin (n \varphi) ).$$

\textbf{23. Как найти комплексные корни $\boldsymbol{n}$-ой степени из комплексного числа? Сделайте эскиз, на котором отметьте исходное число и все корни из него.\\}
Пусть дано комплексное число $w = \rho \left( \cos \psi + i \sin \psi \right)$. Тогда $$\sqrt[n]{w} = \left\{ \sqrt[n]{\rho} \left( \cos \frac{\psi + 2\pi k}{n} + i \sin \frac{\psi + 2\pi k}{n} \right) \mid k = \overline{0, n-1} \right\}.$$
Корни $n$-й степени из $w$ лежат в вершинах правильного $n$-угольника, вписанного в окружность радиуса $\sqrt[n]{\rho}$. Первая точка имеет аргумент $\frac{\psi}{n}$.

\textbf{24. Сформулируйте основную теорему алгебры. Сформулируйте теорему Безу.\\}
\textit{Основная теорема алгебры:} Для любого многочлена $a_n z^n + a_{n-1} z^{n-1} + \hdots + a_1 z + a_0 = 0$, где $\forall i = \overline{0, n}: a_i \in \C,  a_n \neq 0$, существует корень $z_0 \in \C$.\\
\textit{Теорема Безу:} Если $\deg f(x) > 0$, то остаток от деления многочлена $f(x)$ на $(x - c)$ равен $f(c)$.

\textbf{25. Выпишите формулу Эйлера. Выпишите выражения для синуса и косинуса через экспоненту.}
$$e^{i \varphi} = \cos \varphi + i \sin \varphi$$
$$\cos \varphi = \frac{e^{i \varphi} + e^{-i \varphi}}{2}$$
$$\sin \varphi = \frac{e^{i \varphi} - e^{-i \varphi}}{2i}$$
\pagebreak

\textbf{26. Выпишите формулы Виета для многочлена третьей степени.\\}
Пусть $c_1, c_2, c_3$ -- корни многочлена $P_n(x) = 1 \cdot x^3 + a_2 x^2 + a_1 x + a_0$. Тогда:\\
$a_2 = - ( c_1 + c_2 + c_3 ) \\
a_1 = c_1 c_2 + c_2 c_3 + c_1 c_3 \\
a_0 = - c_1 c_2 c_3$

\textbf{27. Какие многочлены называются неприводимыми?\\}
Многочлен $f$ называется неприводимым, если не существует его нетривиального разложения (т.е. не существует многочленов $g$ и $h$, таких что $\deg g < \deg f$ и $\deg h < \deg f$, а $f = g \cdot h$).

\textbf{28. Сформулируйте утверждение о разложении многочленов на неприводимые множители над полем комплексных чисел.\\}
Комплексный многочлен степени $n$ всегда раскладывается над полем комплексных чисел в произведение неприводимых (т.е. многочленов степени 1).

\textbf{29. Какие бинарные операции называются ассоциативными, а какие коммутативными?\\}
Бинарная операция $*$ называется ассоциативной, если $(a * b) * c = a * (b * c)$.\\
Бинарная операция $*$ называется коммутативной, если $a * b= b * a$.

\textbf{30. Дайте определения полугруппы и моноида. Приведите примеры.\\}
Множество с заданной на нём бинарной ассоциативной операцией называется полугруппой.\\
\textit{Пример:} $\left( \N \setminus \left\{ 1 \right\}, \hspace{1mm} \cdot \hspace{1mm} \right)$.\\
Полугруппа, в которой есть нейтральный элемент, называется моноидом.\\
\textit{Пример:} $\left( \N, \hspace{1mm} \cdot \hspace{1mm} \right)$.

\textbf{31. Сформулируйте определение группы. Приведите пример.\\}
Моноид, все элементы которого обратимы, называется группой.\\
\textit{Пример:} $\mathbf{GL}_n ( \R )$.

\textbf{32. Что такое симметрическая группа? Укажите число элементов в ней.\\}
Множество всех подстановок длины $n$ с операцией композиции называется симметрической группой $\mathbf{S}_n$, причём $\left\vert \mathbf{S}_n \right\vert = n!$

\textbf{33. Что такое общая линейная и специальная линейная группы?\\}
Общая линейная группа -- множество всех невырожденных матриц с операцией матричного умножения: $$\mathbf{GL}_n( \R ) = \left( \left\{ A \in \mathbf{M}_n( \R ) \mid \det A \neq 0 \right\}, \hspace{1mm} \cdot \hspace{1mm} \right).$$\\
Специальная линейная группа -- множество матриц с единичным определителем: $$\mathbf{SL}_n( \R ) = \left( \left\{ A \in \mathbf{GL}_n( \R ) \mid \det A = 1 \right\}, \hspace{1mm} \cdot \hspace{1mm} \right).$$

\textbf{34. Сформулируйте определение абелевой группы. Приведите пример.\\}
Группа с коммутативной операцией называется абелевой.\\
\textit{Примеры:} $\left( \mathbb{V}_3, \hspace{1mm} + \right), \left( \Z, \hspace{1mm} + \right)$.

\textbf{35. Дайте определение подгруппы. Приведите пример группы и её подгруппы.\\}
Подмножество $H \subseteq G$ называется подгруппой в $G$, если:\\
1) $e \in H$ ($e$ -- нейтральный элемент из $G$, т.к. он единственен);\\
2) если $h_1, h_2 \in H$, то $h_1 \cdot h_2 \in H$ -- множество $H$ замкнуто относительно умножения;\\
3) если $h \in H$, то $h^{-1} \in H$ -- множество $H$ замкнуто относительно взятия обратного элемента.\\
(Свойства (1)--(3) означают, что $H$ само по себе является группой.)\\
\textit{Пример:} $\mathbf{SL}_n( \R ) \subset \mathbf{GL}_n( \R )$.
\pagebreak

\textbf{36. Дайте определение гомоморфизма групп. Приведите пример.\\}
Пусть даны две группы -- $\left( G_1, * \right)$ и $\left( G_2, \circ \right)$. Тогда отображение $f: G_1 \rightarrow G_2$ называется гомоморфизмом, если выполняется следующее условие: $\forall a, b \in G_1: f(a * b) = f(a) \circ f(b)$.\\
\textit{Пример:} если $G_1 = \mathbf{GL}_n( \R ), G_2 = \R^* = \left( \R \setminus \left\{ 0 \right\}, \hspace{1mm} \cdot \hspace{1mm} \right)$, то отображение $f(A) = \det A$ является гомоморфизмом из $G_1$ в $G_2$.

\textbf{37. Дайте определение изоморфизма групп. Приведите пример.\\}
Биективный гомоморфизм называется изоморфизмом.\\
\textit{Пример:} если $G_1 = \left( \R_+, \hspace{1mm} \cdot \hspace{1mm} \right), G_2 = \left( \R, \hspace{1mm} + \right)$, то отображение $f(x) = \ln x$ является изоморфизмом из $G_1$ в $G_2$.

\textbf{38. Сформулируйте определение циклической группы. Приведите пример.\\}
Пусть $g$ -- элемент группы $G$. Если любой элемент $g \in G$ имеет вид $g = a^n$, где $a \in G, n \in \Z$, то $G$ называют циклической группой и обозначают $G = \left\langle a \right\rangle$.\\
\textit{Пример:} $\left( \Z, \hspace{1mm} + \right)$ -- циклическая группа, порождённая $a = 1$.

\textbf{39. Дайте определение порядка элемента.\\}
Если $q$ -- наименьшее натуральное число, для которого $a^q = e$, где $a \in G$ -- элемент группы, а $e$ -- нейтральный элемент, то $q$ называется порядком элемента $a$. Обозначение: $\Ord (a) = q$.\\
Если такого числа не существует, то говорят об элементе бесконечного порядка: $\Ord (a) = \infty$.

\textbf{40. Сколько существует, с точностью до изоморфизма, циклических групп данного порядка?\\}
Все циклические группы одного порядка изоморфны.

\rule{\linewidth}{0.3mm}

\vspace{1mm}
\begin{center}
\begin{LARGE}
\textsf{3 модуль}
\end{LARGE}
\end{center}
\vspace{1mm}

\textbf{1. Что такое ядро гомоморфизма групп? Приведите пример.\\}
Ядром гомоморфизма $f: G \rightarrow F$ называется множество элементов группы $G$, которые переходят в $e_F$, т.е. в нейтральный элемент группы $F$: $\Ker{f} = \left\{ g \in G \mid f(g) = e_F \right\}$.\\
\textit{Пример:} Если $f(A) = \det A$ -- гомоморфизм из группы $\mathbf{GL}_n( \R )$ в группу $\R^*$, то его ядро -- $\mathbf{SL}_n( \R )$ -- матрицы с определителем, равным единице.

\textbf{2. Сформулируйте утверждение о том, какими могут быть подгруппы группы целых чисел по сложению.\\}
Любая подгруппа в $\left( \Z, + \right)$ имеет вид $k\Z$ (числа, кратные $k$) для некоторого $k \in \N \cup \left\{ 0 \right\}$.

\textbf{3. Дайте определение левого смежного класса по некоторой подгруппе.\\}
Пусть $G$ -- группа, а $H$ -- её подгруппа, и фиксирован $g \in G$. Левым смежным классом элемента $g$ по подгруппе $H$ называется множество $gH = \left\{ g \cdot h \mid h \in H \right\}$. (Аналогично, правый смежный класс -- $Hg = \left\{ h \cdot g \mid h \in H \right\}$.)

\textbf{4. Дайте определение нормальной подгруппы.\\}
Подгруппа $H$ группы $G$ называется нормальной, если $\forall g \in G: gH = Hg$. Обозначение: $H \triangleleft G$.

\textbf{5. Что такое индекс подгруппы?\\}
Индексом подгруппы $H$ в группе $G$ называется количество левых смежных классов $G$ по $H$. Обозначение: $\left[ G : H \right]$

\textbf{6. Сформулируйте теорему Лагранжа.\\}
Пусть $G$ -- конечная группа и $H \subseteq G$ -- её подгруппа. Тогда $\left\vert G \right\vert = \left\vert H \right\vert \cdot \left[ G : H \right]$
\pagebreak

\textbf{7. Сформулируйте критерий нормальности подгруппы, использующий сопряжение.\\}
Пусть $H \subseteq G$. Тогда следующие условия эквивалентны:\\
1) $H \triangleleft G$;\\
2) $\forall g \in G: g H g^{-1} \subseteq H$;\\
3) $\forall g \in G: g H g^{-1} = H$.

\textbf{8. Дайте определение факторгруппы.\\}
Пусть $H$ -- нормальная подгруппа в группе $G$. Тогда $G / H$ -- множество левых смежных классов по $H$ с операцией умножения $(g_1 H) \cdot (g_2 H) = g_1 \cdot g_2 \cdot H$ -- называется факторгруппой.

\textbf{9. Что такое естественный гомоморфизм?\\}
Естественный гомоморфизм $\varepsilon : G \rightarrow G / H$ -- гомоморфизм из группы $G$ в факторгруппу $G$ по $H$, который сопоставляет элементу $a \in G$ его смежный класс.

\textbf{10. Сформулируйте критерий нормальности подгруппы, использующий понятие ядра гомоморфизма.\\}
$H \triangleleft G$ $\Leftrightarrow$ $H = \Ker{f}$, где $f$ -- гомоморфизм из $G$ (куда отображает -- неважно).

\textbf{11. Сформулируйте теорему о гомоморфизме групп. Приведите пример.\\}
Пусть отображение $f: G \rightarrow F$ -- гомоморфизм групп. Тогда образ гомоморфизма, $\Im{f} = \left\{ a \in F \mid \exists g \in G: f(g) = a \right\}$, изоморфен (как группа) факторгруппе $G / \Ker{f}$, т.е. $G / \Ker{f} \cong \Im{f}$.
\textit{Пример:} $\Z / n\Z \cong \Z_n$ посредством гомоморфизма $f(k) = k \mod n$.

\textbf{12. Что такое прямое произведение групп?\\}
Прямым произведением двух групп $G_1$ и $G_2$ называется их декартово произведение как множеств с покомпонентным умножением. Если $\left( G_1, \circ \right)$ и $\left( G_2, * \right)$ -- группы, то $\left( G_1 \times G_2, \star \right)$ -- их прямое произведение, если $\forall g_1, h_1 \in G_1, \forall g_2, h_2 \in G_2: (g_1, g_2) \star (h_1, h_2) = (g_1 \circ h_1, g_2 * h_2)$.

\textbf{13. Сформулируйте определение автоморфизма и внутреннего автоморфизма.\\}
Автоморфизм -- это изоморфизм из $G$ в $G$.\\
Внутренний автоморфизм -- это отображение $I_a: g \mapsto a g a^{-1}$ (сопряжение).

\textbf{14. Что такое центр группы? Приведите пример.\\}
Центр группы -- это подмножество элементов $Z(G) = \left\{ a \in G \mid a \cdot b = b \cdot a \quad \forall b \in G \right\}$, коммутирующих со всеми.\\
\textit{Пример:} $Z(\Q) = \Q$, где $\Q$ -- группа рациональных чисел по сложению.

\textbf{15. Что можно сказать про факторгруппу группы по её центру?\\}
Факторгруппа группы по её центру изоморфная группе всех внутренних автоморфизмов:\\
$G / Z(G) \cong \Inn{G}$.

\textbf{16. Сформулируйте теорему Кэли.\\}
Любая конечная группа порядка $n$ изоморфна некоторой подгруппе группы $\mathbf{S}_n$.

\textbf{17. Дайте определение кольца.\\}
Пусть $K \neq \varnothing$ -- множество, на котором заданы две бинарные операции: сложение и умножение, такие что:\\
1) $\left( K, \hspace{1mm} + \right)$ -- абелева группа;\\
2) $\left( K, \hspace{1mm} \cdot \hspace{1mm} \right)$ -- полугруппа;\\
3) умножение дистрибутивно по сложению:\\
$\forall a, b, c \in K:$\\
$(a + b) \cdot c = a \cdot c + b \cdot c$,\\
$c \cdot (a + b) = c \cdot a + c \cdot b$.\\
Тогда $K$ -- кольцо.
\pagebreak

\textbf{18. Что такое коммутативное кольцо? Приведите примеры коммутативного и некоммутативного колец.\\}
Если $\forall x, y \in K: x \cdot y = y \cdot x$ (т.е. умножение коммутативно), то кольцо $\left( K, +, \hspace{1mm} \cdot \hspace{0.5mm} \right)$ называется коммутативным.\\
\textit{Примеры:} кольцо $\left( \Z, +, \hspace{1mm} \cdot \hspace{0.5mm} \right)$ коммутативно, а кольцо $\left( \mathbf{M}_n( \R ), +, \hspace{1mm} \cdot \hspace{0.5mm} \right)$ некоммутативно.

\textbf{19. Дайте определение делителей нуля.\\}
Если $a \cdot b = 0$ при $a \neq 0$ и $b \neq 0$ в кольце $K$, то $a$ называется левым (а $b$ -- правым) делителем нуля.

\textbf{20. Дайте определение целостного кольца. Приведите пример.\\}
Коммутативное кольцо с единицей, не равной нулю, и без делителей нуля называется целостным кольцом.\\
\textit{Пример:} $\left( \Z, +, \hspace{1mm} \cdot \hspace{0.5mm} \right)$.

\textbf{21. Какие элементы кольца называются обратимыми?\\}
Элемент $a$ коммутативного кольца с единицей называется обратимым, если существует элемент $a^{-1}$, такой что $a \cdot a^{-1} = a^{-1} \cdot a = 1$.

\textbf{22. Дайте определение поля. Приведите три примера.\\}
Поле -- это коммутативное кольцо с единицей, не равной нулю, в котором каждый элемент $a \neq 0$ обратим.\\
\textit{Примеры:} $\Q = \left( \Q, +, \hspace{1mm} \cdot \hspace{0.5mm} \right), \quad \R = \left( \R, +, \hspace{1mm} \cdot \hspace{0.5mm} \right), \quad \C = \left( \C, +, \hspace{1mm} \cdot \hspace{0.5mm} \right)$.

\textbf{23. Дайте определение подполя. Привести пример пары: поле и его подполе.\\}
Подполе -- подмножество поля, которое само является полем относительно тех же операций.\\
\textit{Пример:} $\Q \subset \R \subset \C$.

\textbf{24. Дайте определение характеристики поля. Привести примеры: поля конечной положительной характеристики и поля нулевой характеристики.\\}
Пусть $\F$ -- поле. Характеристикой поля называется наименьшее натуральное число $q$, такое что $\underbrace{1 + 1 + \hdots + 1}_{q} = 0$. Если такого $q$ нет, то характеристика равна нулю. Обозначение: $\Char{\F}$.\\
\textit{Примеры:} $\Char{\Q} = \Char{\R} = \Char{\C} = 0$, $\Char{\Z_p} = p > 0$.

\textbf{25. Сформулируйте утверждение о том, каким будет простое подполе в зависимости от характеристики.\\}
Пусть $P$ -- поле, а $P_0$ -- его простое подполе. Тогда:\\
1) если $\Char{P} = p > 0$, то $P_0 \cong \Z_p$;\\
2) если $\Char{P} = 0$, то $P_0 \cong \Q$.

\textbf{26. Дайте определение идеала. Что такое главный идеал?\\}
Подмножество $I$ кольца $K$ называется (двусторонним) идеалом, если оно:\\
1) является подгруппой $\left( K, + \right)$ по сложению;\\
2) $\forall a \in I, \forall r \in K: r \cdot a \in I$ и $a \cdot r \in I$.\\
Идеал $I$ называется главным, если $\exists a \in K: I = \left\{ r \cdot a \mid r \in K \right\}$. Тогда идеал $I$ порождён $a$.

\textbf{27. Сформулируйте определение гомоморфизма колец.\\}
Отображение $\varphi : K_1 \rightarrow K_2$ -- гомоморфизм колец $K_1$ и $K_2$, если $\forall a, b \in K_1:$\\
1) $\varphi (a + b) = \varphi (a) \oplus \varphi (b)$;\\
2) $\varphi (a \cdot b) = \varphi (a) * \varphi (b)$.

\textbf{28. Сформулируйте теорему о гомоморфизме колец. Приведите пример.\\}
Пусть $K_1$ и $K_2$ -- два кольца, $\varphi : K_1 \rightarrow K_2$ -- гомоморфизм. Тогда: $K_1 / \Ker{\varphi} \cong \Im{\varphi}$.\\
\textit{Пример:} $\Z / \Z_n \cong \left\langle n \right\rangle$.
\pagebreak

\textbf{29. Сформулируйте критерий того, что кольцо вычетов по модулю $\boldsymbol{n}$ является полем.\\}
$\Z_n$ является полем $\Leftrightarrow$ $n$ -- простое.

\textbf{30. Сформулируйте теорему о том, когда факторкольцо кольца многочленов над полем само является полем.\\}
Пусть $\F$ -- поле, а $f(x) \in \F[x]$. Тогда факторкольцо $\F[x] / \left\langle f(x) \right\rangle$ является полем $\Leftrightarrow$ $f(x)$ неприводим над $\F$.

\textbf{31. Дайте определение алгебраического элемента над полем.\\}
Элемент $\alpha \in \mathbb{P}$ называется алгебраическим над подполем $\F \subset \mathbb{P}$, если $\exists f(x) \not\equiv 0: f( \alpha ) = 0$.

\textbf{32. Что такое поле рациональных дробей?\\}
Поле рациональных дробей -- это поле $$\mathcal{P}(x) = \left\{ \frac{f(x)}{g(x)} \mid f(x), g(x) \in \mathcal{P}[x], g(x) \not\equiv 0 \right\}.$$

\textbf{33. Сформулируйте утверждение о том, что любое конечное поле может быть реализовано как факторкольцо кольца многочленов по некоторому идеалу.\\}
Любое конечное поле $\F_{q}$, где $q = p^n$, а $p$ -- простое, можно реализовать в виде $\Z_p[x] / \left\langle h(x) \right\rangle$, где $h(x)$ -- неприводимый многочлен степени $n$ над $\Z_p$.

\textbf{34. Сформулируйте китайскую теорему об остатках (через изоморфизм колец).\\}
Пусть $n \in \Z, n = n_1 \cdot n_2 \cdot \hdots \cdot n_m$, причём $\forall i \forall j:$ $n_i$ и $n_j$ взаимно просты. Тогда $$\Z_n \cong \Z_{n_1} \times \Z_{n_2} \times \hdots \times \Z_{n_m}.$$

\textbf{35. Сформулируйте утверждение о том, сколько элементов может быть в конечном поле.\\}
Число элементов конечного поля всегда $p^n$, где $p$ -- простое, а $n \in \N$.

\textbf{36. Дайте определение линейного (векторного) пространства.\\}
Пусть $\F$ -- поле, $V$ -- произвольное множество, на котором задано 2 операции: сложение и умножение на число (т.е. на элемент из $\F$). Это означает, что $\forall x, y \in V$ существует элемент $x + y \in V$ и $\forall \alpha \in \F: \exists \alpha x \in V$. Множество $V$ называется линейным (векторным) пространством, если выполнены следующие условия:\\
$\forall x, y, z \in V$ и $\forall \lambda, \mu \in \F$:\\
1) $(x + y) + z = x + (y + z)$ -- ассоциативность сложения;\\
2) $\exists 0 \in V: \forall x \in V: x + 0 = 0 + x = x$ -- существует нейтральный элемент по сложению;\\
3) $\forall x \in V: \exists(-x) \in V: x + (-x) = (-x) + x = 0$ -- существует противоположный элемент по сложению;\\
4) $x + y = y + x$ -- коммутативность сложения;\\
5) $\forall x \in V: 1 \cdot x = x \cdot 1 = x$ -- нейтральность $1 \in \F$;\\
6) $\mu( \lambda x ) = ( \mu \lambda )x$ -- ассоциативность умножения на число;\\
7) $( \lambda + \mu )x = \lambda x + \mu x$ -- дистрибутивность умножения относительно сложения чисел;\\
8) $\lambda (x + y) = \lambda x + \lambda y$ -- дистрибутивность умножения относительно сложения векторов.

\textbf{37. Дайте определение базиса линейного (векторного) пространства.\\}
Базисом линейного пространства $V$ называется упорядоченный набор векторов $b_1, \hdots, b_n$, такой что:\\
1) $b_1, \hdots, b_n$ -- линейно независимы;\\
2) любой вектор из $V$ представляется в виде линейной комбинации $b_1, \hdots, b_n$, т.е. $\forall x \in V: x = x_1 b_1 + x_2 b_2 + \hdots + x_n b_n$. При этом $x_1, \hdots, x_n$ -- координаты вектора $x$ в базисе $b_1, \hdots, b_n$.
\pagebreak

\textbf{38. Что такое размерность пространства?\\}
Максимальное количество линейно независимых векторов в данном линейном пространстве $V$ называется размерностью этого линейного пространства. Обозначение: $\Dim{V}$.

\textbf{39. Дайте определение матрицы перехода от старого базиса линейного пространства к новому.\\}
Матрицей перехода от базиса $\A$ к базису $\B$ называется матрица:
$$T_{\A \rightarrow \B} =
\left( \begin{matrix}
t_{11} & t_{12} & \hdots & t_{1n} \\
t_{21} & t_{22} & \hdots & t_{2n} \\
\vdots & \vdots & \ddots & \vdots \\
t_{n1} & t_{n2} & \hdots & t_{nn}
\end{matrix} \right),$$
где в $i$-м столбце стоят коэффициенты разложения вектора $b_i$ по базису $\A$.

\textbf{40. Выпишите формулу для описания изменения координат вектора при изменении базиса.\\}
$x^b = T_{\A \rightarrow \B}^{-1} \cdot x^a$, где $x^a = \left( x_1^a, x_2^a, \hdots, x_n^a \right)^T$, $x^b = \left( x_1^b, x_2^b, \hdots, x_n^b \right)^T$, а $T_{\A \rightarrow \B}$ -- матрица перехода от базиса $\A$ к базису $\B$.

\textbf{41. Дайте определение подпространства в линейном пространстве.\\}
Подмножество $W$ линейного пространства $V$ называется подпространством, если оно само является пространством относительно операций в объемлющем пространстве $V$.

\textbf{42. Дайте определения линейной оболочки конечного набора векторов и ранга системы векторов.\\}
Множество $L(a_1, \hdots, a_k) = \left\{ \lambda_1 a_1 + \hdots + \lambda_k a_k \mid \lambda_i \in \F \right\}$ -- множество всех линейных комбинаций векторов $a_1, \hdots, a_k$ -- называется линейной оболочкой набора (системы) $a_1, \hdots, a_k$.\\
Рангом системы векторов $a_1, \hdots, a_k$ в линейном пространстве называется размерность их линейной оболочки: $\Rg{(a_1, \hdots, a_k)} = \Dim{L(a_1, \hdots, a_k)}$.

\textbf{43. Дайте определения суммы и прямой суммы подпространств.\\}
Множество $H_1 + H_2 = \left\{ x_1 + x_2 \mid x_1 \in H_1 \wedge x_2 \in H_2 \right\}$ называется суммой подпространств $H_1$ и $H_2$.\\
Сумма подпространств $H_1$ и $H_2$ называется прямой и обозначается как $H_1 \oplus H_2$, если $H_1 \cap H_2 = \left\{ 0 \right\}$, т.е. тривиально.

\textbf{44. Сформулируйте утверждение о связи размерности суммы и пересечения подпространств.\\}
Пусть $H_1$ и $H_2$ -- подпространства в $L$. Тогда $\Dim{(H_1 + H_2)} = \Dim{H_1} + \Dim{H_2} - \Dim{H_1 \cap H_2}$.

\textbf{45. Дайте определение билинейной формы.\\}
Пусть $V$ -- линейное пространство над $\R$. Функцию $b: V \times V \rightarrow \R$ называют билинейной формой, если $\forall x, y, z \in V, \forall \alpha, \beta \in \R$ верно, что:\\
1) $b(\alpha x + \beta y, z) = \alpha \cdot b(x, z) + \beta \cdot b(y, z)$\\
2) $b(x, \alpha y + \beta z) = \alpha \cdot b(x, y) + \beta \cdot b(x, z)$\\
Т.е. функция $b$ линейна по каждому из двух аргументов.

\textbf{46. Дайте определение квадратичной формы.\\}
Однородный (т.е. при подстановке вместо переменной $x$ выражения $\alpha x$, за скобку можно вынести $\alpha^k$, где $k$ -- степени однородности) многочлен от $n$ переменных, т.е. $$Q(x) = \sum_{i = 1}^n a_{ii} x_i^2 + 2 \sum_{1 \leq i < j \leq n} a_{ij} x_i x_j \quad (a_{ij} \in \R),$$ называют квадратичной формой.
\pagebreak

\textbf{47. Дайте определения положительной и отрицательной определенности квадратичной формы.\\}
Квадратичную форму $Q(x)$ называют:\\
$\bullet$ положительно определённой, если $\forall x \neq 0: Q(x) > 0$;\\
$\bullet$ отрицательно определённой, если $\forall x \neq 0: Q(x) < 0$.

\textbf{48. Какую квадратичную форму называют знакопеременной?\\}
Квадратичную форму $Q(x)$ называют знакопеременной, если $\exists x, y \in V: Q(x) < 0 < Q(y)$.

\textbf{49. Дайте определения канонического и нормального вида квадратичной формы.\\}
Квадратичную форму $Q(x) = \alpha_1 x_1^2 \hdots + \alpha_n x_n^2$, где $\forall i = \overline{1, n}: a_i \in \R$ (т.е. не имеющую попарных произведений элементов), называют квадратичной формой канонического вида.\\
Если $\forall i = \overline{1, n}: \alpha_i \in \left\{ 0, 1, -1 \right\}$, то такой вид квадратичной формы называют нормальным.

\textbf{50. Как меняется матрица билинейной формы при замене базиса? Как меняется матрица квадратичной формы при замене базиса?\\}
Пусть $B_e$ -- матрица билинейной формы в базисе $\e$, $B_f$ -- в базисе $\f$, а $U$ -- матрица перехода от базиса $\e$ к базису $\f$. Тогда $B_f = U^T \cdot B_e \cdot U$.\\
При переходе от базиса $\e$ к базису $\eprime$ одного и того же линейного пространства $V$ матрица квадратичной формы меняется следующим образом: $A' = S^T \cdot A \cdot S$, где $S$ -- матрица перехода от $\e$ к $\eprime$, а $A$ -- матрица квадратичной формы в базисе $\e$.

\textbf{51. Сформулируйте критерий Сильвестра и его следствие.\\}
\textit{Критерий Сильвестра:} Квадратичная форма $Q(x)$ от $n$ переменных $x = (x_1, \hdots, x_n)$ положительно определена $\Leftrightarrow$ $\Delta_1 > 0 \wedge \Delta_2 > 0 \wedge \hdots \wedge \Delta_n > 0$, где $\Delta_i$ -- главный угловой минор порядка $i$.\\
\textit{Следствие:} $Q(x)$ отрицательно определена $\Leftrightarrow$ $\Delta_1 < 0 \wedge \Delta_2 > 0 \wedge \hdots \wedge (-1)^n \Delta_n > 0$, т.е. знаки главных угловых миноров чередуются, начиная с минуса.

\textbf{52. Сформулируйте закон инерции квадратичных форм. Что такое индексы инерции?\\}
Для любых канонических видов\\
$Q_1(y_1, \hdots, y_m) = \lambda_1 y_1^2 + \hdots + \lambda_m y_m^2 \quad \lambda_i \neq 0, i = \overline{1, m}$,\\
$Q_2(z_1, \hdots, z_k) = \mu_1 z_1^2 + \hdots + \mu_k z_k^2 \quad \mu_j \neq 0, j = \overline{1, k}$\\
одной и той же квадратичной формы выполнено:\\
1) $m = k =$ рангу квадратичной формы;\\
2) $i_+ =$ количество положительных $\lambda_i =$ количество положительных $\mu_j$;\\
3) $i_- =$ количество отрицательных $\lambda_i =$ количество отрицательных $\mu_j$.\\
Числа $i_+$ и $i_-$ называют соответственно положительным и отрицательным индексом инерции.

\textbf{53. Дайте определение линейного отображения. Приведите пример.\\}
Пусть $V_1$ и $V_2$ -- (конечномерные) линейные пространства над полем $\F$. Отображение $\varphi : V_1 \rightarrow V_2$ называется линейным, если:\\
1) $\forall x, y \in V_1: \varphi (x + y) = \varphi (x) + \varphi (y)$;\\
2) $\forall x \in V_1, \forall \alpha \in \F: \varphi (\alpha x) = \alpha \varphi (x).$\\
\textit{Пример:} отображение $\varphi$, поворачивающее векторы из $\mathbb{V}_2$ на заданный угол $\theta$ -- линейное.

\textbf{54. Дайте определение матрицы линейного отображения.\\}
Матрица линейного отображения -- это матрица $$A_{\e\f} =
\left( \begin{matrix}
a_{11} & a_{12} & \hdots & a_{1n} \\
a_{21} & a_{22} & \hdots & a_{2n} \\
\vdots & \vdots & \ddots & \vdots \\
a_{m1} & a_{m2} & \hdots & a_{mn} \\
\end{matrix} \right),$$
где по столбцам стоят координаты образов векторов базиса $\e$ пространства $V_1$ в базисе $\f$ пространства $V_2$.
\pagebreak

\textbf{55. Выпишите формулу для преобразования матрицы линейного отображения при замене базисов. Как выглядит формула в случае линейного оператора?\\}
Пусть $\varphi$ -- линейное отображение из линейного пространства $V_1$ в линейное пространство $V_2$. Пусть $A_{\E_1 \E_2}$ -- матрица линейного отображения в паре базисов: $\E_1$ -- базис в $V_1$, $\E_2$ -- базис в $V_2$, и пусть даны две матрицы перехода: $T_1$ -- матрица перехода от $\E_1$ к $\E'_1$ -- в $V_1$,  $T_2$ -- от $\E_2$ к $\E'_2$ -- в $V_2$. Тогда $A_{\E'_1 \E'_2} = T_2^{-1} \cdot A_{\E_1 \E_2} \cdot T_1$.\\
В случае линейного оператора $\E_1 = \E_2 = \E$ и $\E'_1 = \E'_2 = \E'$, и формула принимает вид $A_{\E'} = T^{-1} \cdot A_{\E} \cdot T$.

\rule{\linewidth}{0.3mm}

\vspace{1mm}
\begin{center}
\begin{LARGE}
\textsf{4 модуль}
\end{LARGE}
\end{center}
\vspace{1mm}

\textbf{1. Сформулируйте утверждение о связи размерностей ядра и образа линейного отображения.\\}
Пусть $\varphi$ -- линейное отображение из $V_1$ в $V_2$. Тогда $\Dim{\Ker{\varphi}} + \Dim{\Im{\varphi}} = \Dim{V_1}$.

\textbf{2. Дайте определения собственного вектора и собственного значения линейного оператора.\\}
Число $\lambda$ называется собственным числом (значением) линейного оператора $\varphi : V \rightarrow V$, если существует ненулевой вектор $x \in V$, такой что $\varphi (x) = \lambda x$. При этом вектор $x$ называется собственным вектором, отвечающим собственному значению $\lambda$.

\textbf{3. Дайте определения характеристического уравнения и характеристического многочлена квадратной матрицы.\\}
Для произвольной квадратной матрицы $A$ определитель $\chi_A ( \lambda ) = \det (A - \lambda E)$ называют характеристическим многочленом матрицы $A$, а выражение $\chi_A ( \lambda ) = 0$ -- характеристическим уравнением.

\textbf{4. Сформулируйте утверждение о связи характеристического уравнения и спектра линейного оператора.\\}
$\lambda$ -- собственное значение линейного оператора $A$ $\Leftrightarrow$ $\lambda$ -- корень характеристического уравнения.

\textbf{5. Дайте определение собственного подпространства.\\}
Пусть $A: V \rightarrow V$ -- линейный оператор, $\lambda$ -- его собственное значение. Тогда множество $V_\lambda = \left\{ x \in V \mid Ax = \lambda x \right\}$ -- собственное подпространство, отвечающее $\lambda$.

\textbf{6. Дайте определения алгебраической и геометрической кратности собственного значения. Какое неравенство их связывает?\\}
Алгебраической кратностью собственного значения $\lambda$ называется кратность $\lambda$ как корня характеристического уравнения.\\
Геометрической кратностью называется размерность собственного подпространства $V_\lambda$, отвечающего данному собственному значению: $\Dim{V_\lambda} = \Dim{\Ker{(A - \lambda E)}}$.\\
Всегда верно, что $1 \leq $ геометрическая кратность $\lambda$ $\leq$ алгебраическая кратность $\lambda$.

\textbf{7. Каким свойством обладают собственные векторы линейного оператора, отвечающие различным собственным значениям?\\}
Пусть $\lambda_1, \hdots, \lambda_k$ -- собственные значения линейного оператора $A$, причём $\forall i \forall j \neq i: \lambda_i \neq \lambda_j$, а $v_1, \hdots, v_k$ -- соответствующие собственные векторы. Тогда $v_1, \hdots, v_k$ линейно независимы.

\textbf{8. Сформулируйте критерий диагональности матрицы оператора.\\}
Матрица линейного оператора является диагональной в данном базисе $\Leftrightarrow$ все векторы базиса являются собственными для данного линейного оператора.
\pagebreak

\textbf{9. Сформулируйте критерий диагонализируемости матрицы оператора с использованием понятия геометрической кратности.\\}
Матрица линейного оператора диагонализируема $\Leftrightarrow$ алгебраическая кратность любого собственного значения равна геометрической.

\textbf{10. Дайте определение жордановой клетки. Сформулируйте теорему о жордановой нормальной форме матрицы оператора.\\}
Матрица вида
$$J_m( \lambda_i ) = \left( \begin{matrix}
\lambda_i & 1 & 0 & \hdots & 0 & 0 \\
0 & \lambda_i & 1 & \hdots & 0 & 0 \\
0 & 0 & \lambda_i & \hdots & 0 & 0 \\
\vdots & \vdots & \vdots & \ddots & \vdots & \vdots \\
0 & 0 & 0 & \hdots & \lambda_i & 1 \\
0 & 0 & 0 & \hdots & 0 & \lambda_i
\end{matrix} \right)$$
называется жордановой клеткой размера $m$. Здесь $\lambda_i$ -- собственное значение линейного оператора.\\
\textit{Теорема о ЖНФ:} Любая матрица $A \in \mathbf{M}_n( \F )$ приводится заменой базиса к жордановой нормальной форме над алгебраически замкнутым полем. (Т.е. $\forall A \in \mathbf{M}_n( \F ): \exists C \in \mathbf{M}_n( \F ) : \det C \neq 0 : A = C \cdot J \cdot C^{-1}$, где $J$ -- ЖНФ.)

\textbf{11. Выпишите формулу для количества жордановых клеток заданного размера.\\}
Для каждого собственного значения $\lambda_i$ количество жордановых клеток размера $h \times h$ с $\lambda_i$ на диагонали равно $q_h = r_{h+1} - 2 r_h + r_{h-1}$, где $r_k = \Rg{\left((A - \lambda_i E)^k\right)}$.

\textbf{12. Сформулируйте теорему Гамильтона--Кэли.\\}
Пусть $\chi_A$ -- характеристический многочлен, $A_e$ -- матрица линейного оператора. Тогда $\chi_A ( A_e ) = 0$.

\textbf{13. Дайте определение корневого подпространства.\\}
Корневое подпространство, отвечающее собственному значению $\lambda_i$ -- это $K_i = \Ker{\left((A - \lambda_i E)^{m_i}\right)}$, где $m_i$ -- алгебраическая кратность собственного значения $\lambda_i$.

\textbf{14. Дайте определение минимального многочлена линейного оператора.\\}
Для матрицы $A$ многочлен $\mu(x)$ называется минимальным, если:\\
1) $\mu(A) = 0$;\\
2) для любого многочлена $f$, такого что $f(A) = 0$, верно, что $\deg f(x) \geq \deg \mu(x)$.\\
(При этом часто договариваются, что старший коэффициент в $\mu (x)$ равен 1.)

\textbf{15. Дайте определение инвариантного подпространства.\\}
Пусть дан линейный оператор $\varphi : V \rightarrow V$ и подпространство $H \subseteq V$. Тогда $H$ называют инвариантным относительно $\varphi$, если $\forall x \in H: \varphi(x) \in H$.

\textbf{16. Дайте определение евклидова пространства.\\}
Евклидово пространство -- это пара, состоящая из линейного пространства $V$ над $\R$ и функции (скалярного произведения) $g(x, y): V \times V \rightarrow \R$, такой что она:\\
1) $\forall x, y \in V: g(x, y) = g(y, x)$ -- симметрична;\\
2) линейна по каждому аргументу;\\
3) $\forall x \in V (x \neq 0): g(x, x) > 0$ и $g(x, x) = 0$ $\Leftrightarrow$ $x = 0$ -- положительно определена.

\textbf{17. Выпишите неравенства Коши--Буняковского и треугольника.\\}
\textit{Неравенство Коши--Буняковского:} $\forall x, y \in \E$ справедливо неравенство $\left\vert g(x, y) \right\vert \leq \left\| x \right\| \cdot \left\| y \right\|$.\\
\textit{Неравенство треугольника:} $\forall x, y \in \E: \left\| x + y \right\| \leq \left\| x \right\| + \left\| y \right\|$.

\textbf{18. Дайте определения ортогонального и ортонормированного базисов.\\}
Базис, все векторы которого попарно ортогональны, называется ортогональным базисом.\\
Если все векторы базиса ортонормированы, то базис называется ортонормированным.
\pagebreak

\textbf{19. Дайте определение матрицы Грама.\\}
Матрица
$$\Gamma = \left( \begin{matrix}
g(a_1, a_1) & \hdots & g(a_n, a_1) \\
\vdots & \ddots & \vdots \\
g(a_1, a_n) & \hdots & g(a_n, a_n) \\
\end{matrix} \right)$$
скалярного произведения как билинейной формы называется матрицей Грама. Здесь $a_1, \hdots, a_n$ -- векторы базиса, а в $i$-й строке, $j$-м столбце стоит скалярное произведение вектора $a_j$ на вектор $a_i$.

\textbf{20. Выпишите формулу для преобразования матрицы Грама при переходе к новому базису.\\}
Пусть $\Gamma$ и $\Gamma'$ -- матрицы Грама соответственно в базисах $\e$ и $\eprime$, а матрица $U$ -- матрица перехода от базиса $\e$ к базису $\eprime$. Тогда $\Gamma' = U^T \Gamma U$.

\textbf{21. Как меняется определитель матрицы Грама (грамиан) при применении процесса ортогонализации Грама--Шмидта?\\}
Определитель матрицы Грама (грамиан) не изменяется при применении процесса ортогонализации Грама--Шмидта.

\textbf{22. Сформулируйте критерий линейной зависимости с помощью матрицы Грама.\\}
Векторы $a_1, \hdots, a_k \in \E$ линейно независимы $\Leftrightarrow$ $\Gr{a_1, \hdots, a_k} \neq 0$. Здесь $\Gr{a_1, \hdots, a_k}$ -- определитель матрицы Грама для векторов $a_1, \hdots, a_k$. 

\textbf{23. Дайте определение ортогонального дополнения.\\}
Пусть $H$ -- подпространство в линейном пространстве $V$. Тогда ортогональное дополнение к $H$ -- это множество
$$H^\perp = \left\{ x \in V \mid \forall y \in H: (x, y) = 0 \right\}.$$

\textbf{24. Дайте определения ортогональной проекции вектора на подпространство и ортогональной составляющей.\\}
Любой вектор $x \in V$ можно разложить в сумму $x = y + z$, где $y \in H$ -- ортогональная проекция $x$ на $H$, а $z \in H^\perp$ -- ортогональная составляющая $x$ относительно $H$.

\textbf{25. Выпишите формулу для ортогональной проекции вектора на подпространство, заданное как линейная оболочка данного линейно независимого набора векторов.\\}
Пусть подпространство $H$ задано как $L(a_1, \hdots, a_k)$, причём векторы $a_1, \hdots, a_k$ -- линейно независимы. Тогда ортогональная проекция вектора $x$ на $H$ вычисляется как $y = \mathrm{pr}_H x = A (A^T A)^{-1} A^T x$.

\textbf{26. Выпишите формулу для вычисления расстояния с помощью определителей матриц Грама.\\}
Расстояние $\rho(P, M)$ между линейным многообразием $P$ и точкой $M$ (определяемой через радиус-вектор $x$), где $P = x_0 + L(a_1, \hdots, a_k)$, а $a_1, \hdots, a_k$ -- линейно независимы, может быть найдено по формуле $$\rho(P, M) = \sqrt{\frac{\Gr{a_1, \hdots, a_k, x - x_0}}{\Gr{a_1, \hdots, a_k}}}.$$

\textbf{27. Дайте определение сопряжённого оператора в евклидовом пространстве.\\}
Линейный оператор $\A^* : \E \rightarrow \E$ называется сопряжённым к линейному оператору $\A : \E \rightarrow \E$, если $\forall x, y \in \E: (Ax, y) = (x, A^* y)$.

\textbf{28. Дайте определение самосопряжённого (симметрического) оператора.\\}
Линейный оператор $\A$ называется самосопряжённым (симметрическим), если $\forall x, y \in \E: (Ax, y) = (x, Ay)$, т.е. $A = A^*$.
\pagebreak

\textbf{29. Как найти матрицу сопряжённого оператора в произвольном базисе?\\}
Для любого линейного оператора $\A$ в евклидовом пространстве $\E$ существует единтсвенный сопряжённый оператор $\A^* : \E$ $\rightarrow$ $\E$ с матрицей $A^*_\b = \Gamma^{-1} A^T_\b \Gamma$, где $\Gamma$ -- матрица Грама в базисе $\b$.

\textbf{30. Каким свойством обладают собственные значения самосопряжённого оператора?\\}
Все корни характеристического уравнения (т.е. собственные значения) самосопряжённого линейного оператора являются действительными числами.

\textbf{31. Что можно сказать про собственные векторы самосопряжённого оператора, отвечающие разным собственным значениям?\\}
Собственные векторы самосопряжённого линейного оператора, отвечающие различным собственным значениям, ортогональны.

\textbf{32. Сформулируйте определение ортогональной матрицы.\\}
Квадратную матрицу $O$ называют ортогональной, если $O^T \cdot O = E$.

\textbf{33. Сформулируйте определение ортогонального оператора.\\}
Линейный оператор $\A : \E \rightarrow \E$ называется ортогональным, если $\forall x, y \in \E : (Ax, Ay) = (x, y)$.

\textbf{34. Сформулируйте критерий ортогональности оператора, использующий его матрицу.\\}
Матрица $A$ линейного оператора $\A$ в ортонормированном базисе ортогональна $\Leftrightarrow$ $\A$ -- ортогональный оператор.

\textbf{35. Каков канонический вид ортогонального оператора? Сформулируйте теорему Эйлера.\\}
\textit{Канонический вид ортогонального оператора:}
$$A' = \left( \begin{matrix}
\mathbf{A_1} & \hdots &        0       &      0     & \hdots &      0     &      0      & \hdots &      0   \\
    \vdots     & \ddots &     \vdots     &   \vdots   & \ddots &   \vdots   &   \vdots    & \ddots &   \vdots \\
       0       & \hdots & \mathbf{A_k} &      0     & \hdots &      0     &      0      & \hdots &      0   \\
       0       & \hdots &        0       & \mathbf{1} & \hdots &      0     &      0      & \hdots &      0   \\
    \vdots     & \ddots &     \vdots     &   \vdots   & \ddots &   \vdots   &   \vdots    & \ddots &   \vdots \\
       0       & \hdots &        0       &      0     & \hdots & \mathbf{1} &      0      & \hdots &      0   \\
       0       & \hdots &        0       &      0     & \hdots &      0     & \mathbf{-1} & \hdots &      0   \\
    \vdots     & \ddots &     \vdots     &   \vdots   & \ddots &   \vdots   &   \vdots    & \ddots &   \vdots \\
       0       & \hdots &        0       &      0     & \hdots &      0     &      0      & \hdots & \mathbf{-1}    
\end{matrix} \right),$$
где $A_i = \left( \begin{matrix}
\cos \varphi_i & -\sin \varphi_i \\
\sin \varphi_i &  \cos \varphi_i
\end{matrix} \right)$ -- матрица поворота.\\
\textit{Теорема Эйлера:} Любой ортогональный оператор в $\R^3$ может быть приведён к следующему каноническому виду:
$$A' = \left( \begin{matrix}
\cos \varphi_i & -\sin \varphi_i &   0   \\
\sin \varphi_i &  \cos \varphi_i &   0   \\
       0       &         0       & \pm 1
\end{matrix} \right).$$
(Т.е. любой ортогональный оператор является либо поворотом на некоторый угол $\varphi$ вокруг некоторой оси (если $[A']_{33} = 1$), либо композицией такого поворота и отражения (если $[A']_{33} = -1$)).

\textbf{36. Сформулируйте теорему о существовании для самосопряжённого оператора базиса из собственных векторов.\\}
Для любого самосопряжённого линейного оператора $\A$ существует ортонормированный базис, состоящий из его собственных векторов.\\
(В этом базисе матрица оператора диагональна, а на диагонали стоят вещественные собственные значения, повторяющиеся столько раз, какова их алгебраическая кратность.)\pagebreak

\textbf{37. Сформулируйте теорему о приведении квадратичной формы к диагональному виду при помощи ортогональной замены координат.\\}
Любую квадратичную форму можно ортогональным преобразованием привести к каноническому виду.

\textbf{38. Сформулируйте утверждение о QR-разложении.\\}
Пусть $A \in \mathbf{M}_m (\R)$, а её столбцы $A_1, \hdots, A_m$ линейно независимы. Тогда существуют матрицы $Q$ и $R$, такие что $A = QR$, где $Q$ -- ортогональная, а $R$ -- верхнетреугольная матрица. 

\textbf{39. Сформулируйте теорему о сингулярном разложении.\\}
Для любой прямоугольной матрицы $A \in \mathbf{M}_{mn} (\R)$ имеет место следующее разложение:
$$A = V \cdot \Sigma \cdot U^T \text{-- сингулярное разложение},$$
где $U \in \mathbf{O}_n (\R)$ -- ортогональная матрица размера $n \times n$, $V \in \mathbf{O}_m (\R)$ -- ортогональная матрица размера $m \times m$, а $\Sigma \in \mathbf{M}_{mn} (\R)$ -- диагональная матрица с числами $\sigma_i \geq 0$ на диагонали. (Договариваются, что $\sigma_1 \geq \sigma_2 \geq \hdots \geq \sigma_r > 0$, где $r = \Rg{A}$.)

\textbf{40. Сформулируйте утверждение о полярном разложении.\\}
Любой линейный оператор в евклидовом пространстве представляется в виде композиции симметрического и ортогонального: $A = S \cdot U$, где $A$ -- матрица исходного линейного оператора, $S$ -- симметрическая матрица, а $U$ -- ортогональная.

\textbf{41. Что можно сказать про ортогональное дополнение к образу сопряжённого оператора?\\}
Пусть $\A$ -- линейный оператор $\A : V \rightarrow V$. Тогда $\Ker{\A} = \left( \Im{\A^*} \right)^\perp$

\textbf{42. Сформулируйте теорему Фредгольма и альтернативу Фредгольма.\\}
\textit{Теорема Фредгольма:} $Ax = b$ -- совместная СЛАУ $\Leftrightarrow$ вектор $b$ перпендикулярен всем решениям ОСЛАУ $A^T y = 0$.\\
\textit{Альтернатива Фредгольма:} Либо у СЛАУ $Ax = b$ существует единственное решение для любого $b$, либо $A^T y = 0$ имеет ненулевое решение.

\color{gray}
\rule{\linewidth}{0.3mm}

\begin{center}
\textit{Определения 4 модуля, не использованные на коллоквиуме}
\end{center}

\textbf{43. Дайте определение сопряжённого пространства.\\}

\textbf{44. Выпишите формулу для преобразования координат ковектора при переходе к другому базису.\\}

\textbf{45. Дайте определение взаимных базисов.\\}

\textbf{46. Дайте определение биортогонального базиса.\\}

\end{document}